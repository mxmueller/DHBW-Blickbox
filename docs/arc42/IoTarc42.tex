% Options for packages loaded elsewhere
\PassOptionsToPackage{unicode}{hyperref}
\PassOptionsToPackage{hyphens}{url}
%
\documentclass[
]{article}
\usepackage{amsmath,amssymb}
\usepackage{lmodern}
\usepackage{iftex}
\usepackage{graphicx}

\ifPDFTeX
  \usepackage[T1]{fontenc}
  \usepackage[utf8]{inputenc}
  \usepackage{textcomp} % provide euro and other symbols
\else % if luatex or xetex
  \usepackage{unicode-math}
  \defaultfontfeatures{Scale=MatchLowercase}
  \defaultfontfeatures[\rmfamily]{Ligatures=TeX,Scale=1}
\fi
% Use upquote if available, for straight quotes in verbatim environments
\IfFileExists{upquote.sty}{\usepackage{upquote}}{}
\IfFileExists{microtype.sty}{% use microtype if available
  \usepackage[]{microtype}
  \UseMicrotypeSet[protrusion]{basicmath} % disable protrusion for tt fonts
}{}
\makeatletter
\@ifundefined{KOMAClassName}{% if non-KOMA class
  \IfFileExists{parskip.sty}{%
    \usepackage{parskip}
  }{% else
    \setlength{\parindent}{0pt}
    \setlength{\parskip}{6pt plus 2pt minus 1pt}}
}{% if KOMA class
  \KOMAoptions{parskip=half}}
\makeatother
\usepackage{xcolor}
\IfFileExists{xurl.sty}{\usepackage{xurl}}{} % add URL line breaks if available
\IfFileExists{bookmark.sty}{\usepackage{bookmark}}{\usepackage{hyperref}}
\hypersetup{
  pdftitle={arc42 Template},
  hidelinks,
  pdfcreator={LaTeX via pandoc}}
\urlstyle{same} % disable monospaced font for URLs
\usepackage{longtable,booktabs,array}
\usepackage{calc} % for calculating minipage widths
% Correct order of tables after \paragraph or \subparagraph
\usepackage{etoolbox}
\makeatletter
\patchcmd\longtable{\par}{\if@noskipsec\mbox{}\fi\par}{}{}
\makeatother
% Allow footnotes in longtable head/foot
\IfFileExists{footnotehyper.sty}{\usepackage{footnotehyper}}{\usepackage{footnote}}
\makesavenoteenv{longtable}
\usepackage{graphicx}
\makeatletter
\def\maxwidth{\ifdim\Gin@nat@width>\linewidth\linewidth\else\Gin@nat@width\fi}
\def\maxheight{\ifdim\Gin@nat@height>\textheight\textheight\else\Gin@nat@height\fi}
\makeatother
% Scale images if necessary, so that they will not overflow the page
% margins by default, and it is still possible to overwrite the defaults
% using explicit options in \includegraphics[width, height, ...]{}
\setkeys{Gin}{width=\maxwidth,height=\maxheight,keepaspectratio}
% Set default figure placement to htbp
\makeatletter
\def\fps@figure{htbp}
\makeatother
\setlength{\emergencystretch}{3em} % prevent overfull lines
\providecommand{\tightlist}{%
  \setlength{\itemsep}{0pt}\setlength{\parskip}{0pt}}
\setcounter{secnumdepth}{-\maxdimen} % remove section numbering
\ifLuaTeX
  \usepackage{selnolig}  % disable illegal ligatures
\fi

\title{\includegraphics{header.jpg} Template}
\author{}
\date{Januar 2023}

\begin{document}
\maketitle

% DOCUMENT STARTS HERE


\hypertarget{section-introduction-and-goals}{%
\section{Einführung und Ziele}\label{section-introduction-and-goals}}

\hypertarget{_stakeholder}{%
\subsection{Stakeholder}\label{_stakeholder}}
Ein umfassender Überblick über die Stakeholder des Systems ist von entscheidender Bedeutung. Dies bezieht sich auf sämtliche Personen, Rollen oder Organisationen, die entweder die Architektur des Systems kennen sollten oder von dieser überzeugt werden müssen. Zu den Stakeholdern zählen auch jene, die aktiv mit der Architektur oder dem Code arbeiten, beispielsweise indem sie Schnittstellen nutzen. Ebenso gehören Personen dazu, die die Dokumentation der Architektur benötigen, um ihre eigene Arbeit effizient zu gestalten. Darüber hinaus sind Stakeholder involviert, die Entscheidungen über das System und dessen Entwicklung treffen.
Die nachfolgende Analyse zeigt alle Stakeholder gebündelt in ihren Gewichtungen und Beziehungen.  
\begin{figure}
  \centering
  \includegraphics[width=1\textwidth]{./resources/stakeholderanalyse.drawio.png}
  \caption{Stakeholder Analyse und Zusammenfassung in den Unterscheidungen: Einbindungsgrad, Interesse und Einflüsse. Das Entwicklungsteam geht deutlich als am stärksten partizipativ gekennzeichneten Stakeholder hevor. Am repressivsten zeigen sich die Stakeholder in Form der Fußgänger.}
  \label{fig:deine_label}
\end{figure}



\end{document}
